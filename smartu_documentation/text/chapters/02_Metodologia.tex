\chapter{Metodología de trabajo multidisciplinar}
\label{ch:metodologia}

\section{El trabajo en equipo}
Este proyecto multidisciplinar quiere poner en práctica el mismo concepto sobre el que trabaja: \textit{Gestión de equipos de trabajo multidisciplinares para desarrollo de proyectos}. Aquí entran en juego varios conceptos importantes vistos en el apartado \ref{sec:objetivos} de objetivos:

\begin{itemize}
    \item Por un lado nos encontramos con el reto de \textbf{crear} una plataforma para gestionar proyectos multidisciplinares, facilitando la tarea de encontrar a personas para formar equipo y que trabajen en una idea.
    \item Por otro lado, nosotros mismos como equipo, debemos trabajar en la organización para que todo salga adelante y podamos desarrollar dicha plataforma.
\end{itemize}

Esto nos sirve de \textbf{experiencia} sobre cómo es el trabajo en equipo multidisciplinar. Ayudará a encontrar \textbf{debilidades y fortalezas}, y servirá para que los productos que creemos sean mejores y sean de utilidad ante la problemática del trabajo en equipo.\\

En el nacimiento de este equipo, se planteó que debemos funcionar de forma similar a una Startup \cite{startup}. Siguiendo esa lógica, trabajamos sobre una idea innovadora para resolver un problema complejo, empleando la tecnología para llegar a cumplir el objetivo.\\

Sobre el concepto de la plataforma SmartU, nacen una serie de \textbf{líneas de trabajo} que consideramos importantes para lograr un producto completo y útil para su público objetivo, por ello es necesario que se trabaje en ellas de forma correcta.

\begin{description}
    \item[Software] La línea de trabajo donde crear aplicaciones web y móviles que el público objetivo usará con el fin de facilitar la tarea de la creación de proyectos multidisciplinares.
    \item[Marketing] Esta línea de trabajo concentra sus esfuerzos en la búsqueda de formas de promoción adecuadas al público y que garanticen que SmartU llegue al máximo número de personas posibles. Destacamos por ejemplo el uso de las redes institucionales ya existentes para dar difusión, así como la creación de redes sociales propias de la plataforma como forma de acercamiento a los demás.
    \item[Diseño gráfico] La identidad visual es el reto al que se enfrenta esta línea de trabajo. Se ha de encontrar un estilo y diseño diferenciador que además sea agradable al usuario, y que de alguna forma represente los conceptos sobre los que trabajamos, como por ejemplo el concepto principal de ciudades inteligentes.
    \item[Plan de empresa] El plan de empresa tiene como objetivo encontrar una estrategia de negocio que sirva para que esta \textit{startup} sea viable en el futuro.
\end{description}

\subsection{Mi trabajo}
Mi ocupación se centra en la línea de trabajo de \textbf{Software}. Como estudiante de Ingeniería Informática, mi ocupación sería la de crear la plataforma web de SmartU, donde el público objetivo pueda registrarse y subir sus propuestas de proyecto, establecer criterios para encontrar a los estudiantes de las disciplinas que necesitan, y recibir valoraciones y opiniones del resto de usuarios, todo siguiendo en la medida de lo posible un enfoque de red social que haga el uso de la aplicación más ``humano'' y no tanto el de usar una máquina donde publicas tu idea y esperas que te aparezcan candidatos.

\section{Integrantes del equipo}
El equipo de trabajo se ha compuesto de diferentes \textbf{profesores y estudiantes} de diversas disciplinas de la universidad. Podemos ver su nombre y ocupación en la \textbf{tabla \ref{miembrossmartu}}. Todos hemos aportado conocimientos de nuestro campo al proyecto para intentar lograr los mejores resultados, recibiendo opiniones y puntos de vista muy diferentes que ayudan y complementan.\\

\begin{table}
    \begin{center}
        \begin{tabular}{|p{4.5cm}|p{6.5cm}|}
            \hline
                \rowcolor{Gray}\multicolumn{1}{|c|}{\textbf{Integrante}}
                & \multicolumn{1}{|c|}{\textbf{Ocupación}} \\
            \hline
                Juan Árbol Gutiérrez & Estudiante de CC.EE. y Empresariales \\
            \hline
                Irene Castillo Pardo & Estudiante de Comunicación y Audiovisuales \\
            \hline
                Emilio Chica Jiménez & Estudiante de Ingeniería Informática \\
            \hline
                Victoria Guerra Molina & Estudiante de Comunicación y Audiovisuales \\
            \hline
                Juan José Jiménez García & Estudiante de Ingeniería Informática \\
            \hline
                Javier Labrat Rodríguez & Estudiante de Ingeniería Informática \\
            \hline
                Germán Zayas Cabrera & Estudiante de Bellas Artes \\
            \hline
                Miguel Gea Megías & Profesor de Ingeniería Informática \\
            \hline
                Guillermo Maraver Tarifa & Profesor de CC.EE. y Empresariales \\
            \hline
                Alejandro Grindlay Moreno & Profesor de Ingeniería Civil \\
            \hline
        \end{tabular}
        \caption{Integrantes del proyecto multidisciplinar SmartU}
        \label{miembrossmartu}
    \end{center}
\end{table}

El equipo no se componía de estas personas exactas al comienzo. A lo largo del curso \textbf{fueron incorporándose} más compañeros, lo cual también suponía un reto, por el hecho de gestionar ese cambio no previsto para adaptar todo.\\

Cada uno de los miembros del equipo tenía asignado un rol o tarea. Era importante que cada uno se dedicase a algo relacionado con sus conocimientos, de esta manera se esperaba conseguir mejores resultados, al \textbf{delegar en la persona adecuada la tarea adecuada}. Podemos ver el reparto de roles en la \textbf{tabla \ref{rolessmartu}}.\\

\begin{table}
    \begin{center}
        \begin{tabular}{|p{5cm}|p{6cm}|}
            \hline
                \rowcolor{Gray}\multicolumn{1}{|c|}{\textbf{Integrante/s}}
                & \multicolumn{1}{|c|}{\textbf{Rol o tarea}} \\
            \hline
                Juan Árbol Gutiérrez & Emprendedor y gestor de estrategia empresarial \\
            \hline
                Juan José Jiménez García & Gestor de proyecto y desarrollador de software \\
            \hline
                Emilio Chica Jiménez & Gestor tecnológico y desarrollador de software \\
            \hline
                Irene Castillo Pardo y Victoria Guerra Molina & Gestoras de audiovisuales \\
            \hline
                Germán Zayas Cabrera & Diseñador gráfico \\
            \hline
                Javier Labrat Rodríguez, Miguel Gea Megías, Guillermo Maraver Tarifa y Alejandro Grindlay Moreno & Tutores y consultores para dudas \\
            \hline
        \end{tabular}

        \caption{Integrantes del proyecto multidisciplinar SmartU}
        \label{rolessmartu}
    \end{center}
\end{table}

En la \textbf{tabla \ref{tareassmartu}} se puede ver un resumen general del trabajo realizado por cada uno de los miembros del equipo, o que va a realizar más adelante.

\newpage
\begin{longtable}{|m{4.5cm}|m{6.5cm}|}
    \hline
        \rowcolor{Gray}\multicolumn{1}{|c|}{\textbf{Integrante}}
        & \multicolumn{1}{|c|}{\textbf{Aportaciones}} \\
    \hline
        Juan Árbol Gutiérrez & \begin{itemize}
            \item Organizador del Design Thinking y Brainstorming
            \item Creador del plan estratégico de empresa
            \item Responsable de realización de entrevistas a posibles usuarios objetivo del sistema
            \item Presentador del proyecto en la Facultad de Ciencias de la Actividad Física y del Deporte
        \end{itemize} \\
    \hline
        Irene Castillo Pardo & \begin{itemize}
            \item Colaborador en el Design Thinking y Brainstorming
            \item Creadora del video de presentación del proyecto
            \item Creación del making of del proyecto
        \end{itemize} \\
    \hline
        Emilio Chica Jiménez & \begin{itemize}
            \item Creador de la aplicación móvil del proyecto
            \item Organizador del Design Thinking y Brainstorming
            \item Moderador de seminario de tecnologías emergentes
            \item Gestor de reuniones
        \end{itemize} \\
    \hline
        Victoria Guerra Molina & \begin{itemize}
            \item Colaborador en el Design Thinking y Brainstorming
            \item Investigadora de una campaña de difusión del proyecto en redes sociales y medios de publicidad
        \end{itemize} \\
    \hline
        Juan José Jiménez García & \begin{itemize}
            \item Organizador del Design Thinking y Brainstorming
            \item Gestor del proyecto
            \item Creador de la aplicación web
            \item Coordinador y documentador de reuniones
            \item Coordinador del repositorio de archivos y calendario
        \end{itemize} \\
    \hline
        Javier Labrat Rodríguez & \begin{itemize}
            \item Presentador del proyecto en la Facultad de Ciencias de la Actividad Física y del Deporte
            \item Creador de un proyecto que servirá de muestra para nuestro sistema SmartU
        \end{itemize} \\
    \hline
        Germán Zayas Cabrera & \begin{itemize}
            \item Colaborador en el Design Thinking y Brainstorming
            \item Creador de la identidad visual del proyecto
            \item Creador del diseño de la página web de presentación del proyecto
        \end{itemize} \\
    \hline
        Miguel Gea Megías & \begin{itemize}
            \item Creador de la idea original
            \item Agrupador de los miembros del equipo
            \item Consejero para el desarrollo del proyecto
            \item Colaborador en el Design Thinking y Brainstorming
            \item Gestor de las reuniones
        \end{itemize} \\
    \hline
        Guillermo Maraver Tarifa & \begin{itemize}
            \item Colaborador en el Design Thinking y Brainstorming
            \item Consejero de Juan para el desarrollo del proyecto
            \item Consejero de marketing y promoción del proyecto
        \end{itemize} \\
    \hline
        Alejandro Grindlay Moreno & \begin{itemize}
            \item Colaborador en el Design Thinking y Brainstorming
            \item Consejero para el desarrollo del proyecto
        \end{itemize} \\
    \hline
    \caption{Aportaciones de los miembros del equipo}\label{tareassmartu}\\
\end{longtable}

\subsection{Dependencias entre miembros del equipo}
Una de las cosas que tuvimos en cuenta al principio fue que, en un proyecto multidisciplinar, es muy importante \textbf{encontrar sinergias} entre los miembros del mismo, es decir, encontrar puntos comunes que permitan una colaboración fructuosa entre dos personas. Así, el trabajo que teníamos que hacer cada uno se vería más reforzado al contar con el punto de vista y la ayuda proporcionada por otra persona de diferente disciplina.\\

Así, nos encontramos con los siguientes ejemplos de tareas en las que hemos encontrado sinergias (es decir, que gracias a la colaboración mutua, creemos que el resultado obtenido es mejor que haberlo hecho sin su ayuda):

\begin{itemize}
    \item La propia \textbf{implementación de las plataformas} colaborando con mi compañero Emilio consiguió que el software propuesto presentara una funcionalidad más rica y completa que haberlo hecho por separado.
    \item Germán aportó sinergias en casi todas las líneas de trabajo, ya que su tarea (\textbf{diseño de la identidad corporativa}) es aplicable no solo al software, sino también al marketing.
    \item Entre Germán y Juan hubo colaboración para confeccionar el diseño de la \textbf{página web de presentación del proyecto}, así como la creación de sus contenidos.
    \item Javier, con su proyecto ya terminado, sirvió como \textbf{prueba de concepto}, al permitir incorporar su proyecto al futuro sistema que se va a diseñar para demostrar al público objetivo las capacidades de SmartU.
\end{itemize}

Pero también es cierto que hubo ciertos problemas a lo largo del curso que impidieron la aprición de más sinergias. Principalmente los problemas fueron la falta de tiempo para que algunos miembros pudieran hacer sus tareas, y la tardía definición de todos los conceptos del proyecto y la necesidad de la aplicación móvil, que en un principio no quedaba clara cual podía ser tu utilidad y diferenciación.\\

Esta experiencia nos hace ver que este tipo de proyectos requieren de más dedicación de la que se pensaba. Al ser una primera experiencia piloto, no teníamos del todo claro lo que podía pasar, pero ello nos servirá para que en los años siguientes el proceso mejore.

\section{Gestión del trabajo}
Como encargado de gestionar el proyecto además de realizar software, junto con Miguel llevamos a cabo la \textbf{organización del trabajo} con el objetivo de conseguir la mayor agilidad posible para la comunicación y para documentar los progresos. En las siguientes secciones de este apartado se describe con más detalle lo que se ha realizado para cada uno de los puntos de gestión.

\subsection{Comunicación}
Un proyecto de esta índole necesitaba de varias \textbf{reuniones}, donde poder debatir asuntos y tomar decisiones, así como para realizar ciertas técnicas creativas de generación de ideas y establecer las funcionalidades de los productos a desarrollar. Estos puntos son tratados más adelante en esta documentación.\\

Las reuniones tuvieron lugar en diferentes puntos de encuentro de todos los campus de la Universidad de Granada, siendo el más habitual el de la Facultad de Ciencias Económicas y Empresariales, debido a la disponibilidad de mejores salas de reunión y trabajo en equipo.\\

Debido a la diferencia de disciplinas que presentaban los integrantes, se hacía necesario establecer un \textbf{horario semanal} donde se cuadrase la disponibilidad de todos los miembros para que estuviera el máximo número posible en las reuniones. Dicho horario se actualizaba semana a semana según las necesidades de los integrantes.

Podemos describir el \textbf{proceso de organización de reuniones} de la siguiente manera:

\begin{enumerate}
    \item Como gestión previa a una reunión, ocurría lo siguiente:
    \begin{enumerate}
        \item Se \textbf{comunicaba a los gestores} la necesidad de realizar una reunión para tratar un asunto importante del proyecto.
        \item Los gestores se \textbf{reunían} para establecer los puntos de la próxima reunión.
        \item Se \textbf{comunicaba al resto del equipo} de la necesidad de realizar la reunión y se les pedía que indicasen en el horario los días que mejor le convenían, para encontrar el punto común donde pudieran ir todos.
        \item Los profesores, debido a la posibilidad que presentan de \textbf{reservar salas de reuniones} más adecuadas que los que los estudiantes tenían permitido, se encargaban de encontrar sitio para celebrar la reunión en base a la disponibilidad de los asistentes.\\
        También se encargaban de asegurarse de que la sala que se fuera a reservar contase con material adecuado para la reunión, como un \textbf{proyector, pizarra}, etc.
        \item Se establecía en el calendario oficial del proyecto (con \textbf{Google Calendar} \cite{googlecalendar}) y se informaba al resto del equipo por el medio correspondiente.
    \end{enumerate}
    \item Al comenzar la reunión, se resumían los puntos tratados en la \textbf{reunión anterior}, a modo de recordatorio y para aquellos miembros que no hubieran podido asistir a dicha reunión. Tras el resumen, se procede a informar de los puntos a debatir en la reunión actual.
    \item Al acabar la reunión se comentan los \textbf{nuevos puntos a tratar} de cara a la próxima reunión, y se anotan para no olvidarlos y que los gestores puedan prepararlos.
    \item El gestor del proyecto crea un \textbf{acta de reunión} donde se detallan los puntos debatidos y se apuntan los temas a tratar para la próxima reunión.
\end{enumerate}

Como detalle final de la gestión de la comunicación, en la tabla \ref{canalescomunicacion} se puede ver las valoraciones dadas tras su uso a los distintos canales de comunicación que utilizamos entre los miembros del equipo.

\begin{table}
    \begin{center}
        \begin{tabular}{|l|p{3cm}|p{5cm}|}
            \hline
                \rowcolor{Gray}\multicolumn{1}{|c|}{\textbf{Canal}}
                & \multicolumn{1}{|c|}{\textbf{Utilidad}} & \multicolumn{1}{|c|}{\textbf{Valoración}} \\
            \hline
                E-mail & Usado en las primeras semanas & No es recomendable, debido a que no siempre se comprueba el correo electrónico con regularidad y no se sabe si alguien lo ha leido. \\
            \hline
                WhatsApp \cite{whatsapp} & Utilizado durante todo el proyecto & Aplicación de mensajería más utilizada, respuesta casi inmediata y mayor rapidez de comunicación. \\
            \hline
                Slack \cite{slack} & Utilizado al final del proyecto & Es una plataforma muy utilizada por empresas y equipos de desarrollo, con numerosas funcionalidades de comunicación entre miembros que agilizan bastante la comunicación. \\
            \hline
        \end{tabular}

        \caption{Canales de comunicación empleados y valoración de los mismos}
        \label{canalescomunicacion}
    \end{center}
\end{table}

\subsection{Documentación}
Para gestionar la documentación y todo el contenido que el equipo fuese produciendo a lo largo del tiempo, se hacía necesario contar con un espacio común de alojamiento de archivos. Para esto se optó por utilizar \textbf{Google Drive} \cite{googledrive}, un servicio de almacenamiento de archivos en la nube rápido y sencillo de utilizar que todos los miembros del equipo conocían.\\

Todos tenían acceso a dicho directorio online, y ahí se iban dejando los documentos que generábamos, como pueden ser:
\begin{itemize}
    \item Presentaciones para exponer en reuniones.
    \item Actas de reuniones.
    \item Bocetos de diseño de aplicaciones o de identidad visual.
    \item Calendario semanal actualizado.
\end{itemize}

\section{Resultados}
Tras la realización de los \textbf{seminarios de tecnologías emergentes y técnicas de creatividad}, se procedió a realizar una sesión de \textit{Brainstorming} y un proceso creativo de \textit{Design Thinking} para que todos aportaran ideas y conceptos que nos servirían para dar forma a nuestro sistema a desarrollar. En el anexo \ref{sec:designthinking} podrá encontrar una \textbf{lista de todas las ideas} que se obtuvieron, clasificadas de la siguiente manera:

\begin{itemize}
    \item Ideas
    \item Objetivos
    \item Limitaciones
    \item Aspectos positivos
\end{itemize}

Más adelante se realizó otra sesión de Design Thinking debido a que hubo algunas ausencias en la primera reunión, y sirvió para \textbf{afianzar todo lo visto} en la primera y proceder con ello al desarrollo del producto software, que se detalla en el capítulo \ref{ch:desarrollo}.\\

En el anexo \ref{sec:actas} puede encontrar disponible las actas de reunión realizadas durante este primer año de vida del proyecto.
