\chapter{Conclusiones}
\label{ch:conclusiones}
Este capítulo recoge los resultados obtenidos de la realización del proyecto, siguiendo las directrices y procedimientos marcados en el capítulo \ref{ch:metodologia}. Tras ello, se realiza una valoración de lo que se ha obtenido, y se completa con una serie de sugerencias de mejora de cara a futuras iteraciones del proyecto.

\section{Desarrollo de la metodología}
\subsection{Integrantes del equipo}
El equipo no se componía de estas personas exactas al comienzo. A lo largo del curso \textbf{fueron incorporándose} más compañeros, lo cual también suponía un reto de gestión, por el hecho de gestionar ese cambio no previsto para adaptar todo.\\

\subsection{Design Thinking}
Tras la realización de los \textbf{seminarios de tecnologías emergentes y técnicas de creatividad}, se procedió a realizar una sesión de \textit{Brainstorming} y un proceso creativo de \textit{Design Thinking} para que todos aportaran ideas y conceptos que nos servirían para dar forma a nuestro sistema a desarrollar. En el anexo \ref{sec:designthinking} podrá encontrar una \textbf{lista de todas las ideas} que se obtuvieron, clasificadas de la siguiente manera:

\begin{itemize}
    \item Ideas
    \item Objetivos
    \item Limitaciones
    \item Aspectos positivos
\end{itemize}

Más adelante se realizó otra sesión de Design Thinking debido a que hubo algunas ausencias en la primera reunión, y sirvió para \textbf{afianzar todo lo visto} en la primera y proceder con ello al desarrollo del producto software, que se detalla en el capítulo \ref{ch:desarrollo}.\\

En el anexo \ref{sec:actas} puede encontrar disponible las actas de reunión realizadas durante este primer año de vida del proyecto.

\subsection{Comunicación}
Como detalle final de la gestión de la comunicación, en la tabla \ref{canalescomunicacion} se puede ver las valoraciones dadas tras su uso a los distintos canales de comunicación que utilizamos entre los miembros del equipo.

\begin{table}
    \begin{center}
        \begin{tabular}{|l|p{3cm}|p{5cm}|}
            \hline
                \rowcolor{Gray}\multicolumn{1}{|c|}{\textbf{Canal}}
                & \multicolumn{1}{|c|}{\textbf{Utilidad}} & \multicolumn{1}{|c|}{\textbf{Valoración}} \\
            \hline
                E-mail & Usado en las primeras semanas & No es recomendable, debido a que no siempre se comprueba el correo electrónico con regularidad y no se sabe si alguien lo ha leido. \\
            \hline
                WhatsApp \cite{whatsapp} & Utilizado durante todo el proyecto & Aplicación de mensajería más utilizada, respuesta casi inmediata y mayor rapidez de comunicación. \\
            \hline
                Slack \cite{slack} & Utilizado al final del proyecto & Es una plataforma muy utilizada por empresas y equipos de desarrollo, con numerosas funcionalidades de comunicación entre miembros que agilizan bastante la comunicación. \\
            \hline
        \end{tabular}

        \caption{Canales de comunicación empleados y valoración de los mismos}
        \label{canalescomunicacion}
    \end{center}
\end{table}

\subsection{Sinergias}
Nos encontramos con los siguientes ejemplos de tareas en las que hemos encontrado sinergias (es decir, que gracias a la colaboración mutua, creemos que el resultado obtenido es mejor que haberlo hecho sin su ayuda):

\begin{itemize}
    \item La propia \textbf{implementación de las plataformas} colaborando con mi compañero Emilio consiguió que el software propuesto presentara una funcionalidad más rica y completa que haberlo hecho por separado.
    \item Germán aportó sinergias en casi todas las líneas de trabajo, ya que su tarea (\textbf{diseño de la identidad corporativa}) es aplicable no solo al software, sino también al marketing.
    \item Entre Germán y Juan hubo colaboración para confeccionar el diseño de la \textbf{página web de presentación del proyecto}, así como la creación de sus contenidos.
    \item Javier, con su proyecto ya terminado, sirvió como \textbf{prueba de concepto}, al permitir incorporar su proyecto al futuro sistema que se va a diseñar para demostrar al público objetivo las capacidades de SmartU.
\end{itemize}

Pero también es cierto que hubo ciertos problemas a lo largo del curso que impidieron la aprición de más sinergias. Principalmente los problemas fueron la falta de tiempo para que algunos miembros pudieran hacer sus tareas, y la tardía definición de todos los conceptos del proyecto y la necesidad de la aplicación móvil, que en un principio no quedaba clara cual podía ser tu utilidad y diferenciación.\\

Esta experiencia nos hace ver que este tipo de proyectos requieren de más dedicación de la que se pensaba. Al ser una primera experiencia piloto, no teníamos del todo claro lo que podía pasar, pero ello nos servirá para que en los años siguientes el proceso mejore.

\section{Valoración personal}
Tras todos estos meses de trabajo, la experiencia adquirida es de enorme valor. Aunque ha habido diversos contratiempos, no se puede negar que se han hecho importantes progresos en el inicio de este proyecto. Se ha conseguido formar un equipo de trabajo de diferentes especialidades y se ha podido obtener de todos ellos mucha información, además de aprender a aprovechar los puntos fuertes de cada uno para realizar partes de este proyecto.\\

Personalmente, quiero destacar que ha sido muy revelador el haber compartido trabajo con personas de otras disciplinas y estudios de la universidad. Tras muchos años trabajando en equipo con compañeros informáticos, nos acostumbramos demasiado y no sabemos tratar con personas que no tienen los mismos conocimientos que nosotros. Junto a esto, ha sido positivo el haber podido coordinar en la medida de lo posible a todos para poder reunirnos sin alterar la rutina y obligaciones diarias de cada uno de los miembros.\\

Hemos conseguido llevar a cabo una primera versión de una metodología de trabajo en equipo que confiamos en que mejore con los años. A modo de ``experimento'', hemos aplicado nuestros actuales conocimientos y hemos visto las fortalezas y debilidades de esta forma de trabajar.\\

Se ha conseguido crear dos productos que pueden servir de apoyo para continuar este proyecto, como son la aplicación móvil de mi compañero Emilio y la aplicación web que un servidor ha desarrollado. Es cierto que ha habido ciertas ``complicaciones'' debido a imprevistos y falta de tiempo, y que no he podido crear un producto con un 100\% de calidad, pero la idea de un desarrollo ágil es ir mejorando con el tiempo lo existente e incorporarle nuevas funcionalidades. Por ello estoy muy satisfecho con lo que se ha logrado, tanto a nivel de equipo como a nivel personal.\\

El trabajo en equipo nunca es fácil, ya que requiere de un gran compromiso por parte de todos los integrantes para que pueda haber un cierto nivel de éxito. No importa que sea poco al principio si se consigue poner la primera piedra de un proyecto que se espera que a largo plazo se refine más. El hecho de poder coordinar (en mayor o menor medida) a estudiantes de diferentes disciplinas de conocimiento pone en valor lo enriquecedor que supone un trabajo que recibe apoyo de distintos puntos de vista.\\

Por ello, espero que estas páginas y las de los proyectos del resto de mis compañeros sean en el futuro de gran utilidad y permitan que en el futuro, los proyectos multidisciplinares sean una parte más dentro de la vida universitaria, y que los estudiantes no tengan miedo a adentrarse en un proyecto en equipo. Nadie dijo que fuese algo fácil, pero no es imposible.

\section{Mejoras para el futuro}
El proyecto cuenta con diversas líneas de trabajo, en las que todavía queda espacio para mejoras y ampliación de su desarrollo. En un principio se supo que no iban a quedar todas las líneas finalizadas en su primer año de vida, así que las personas que decidan continuarlo se encargarán de seguir completándolo.

El equipo multidisciplinar recomienda que se sigan las directrices mencionadas en el capítulo de la metodología de trabajo, y animamos a que se perfeccione y corrija todo lo que se vea que requiere mejora, para conseguir mejores resultados en años venideros.\\

Existen numerosas mejoras que se pueden realizar en lo que respecta a mi línea de trabajo. Para consultar las posibles mejoras de otras líneas, consulta los TFGs de mis compañeros de equipo multidisciplinar. He confeccionado la siguiente lista en base a mi experiencia y trabajo realizado:

\begin{itemize}
    \item Debido a problemas de tiempo, mi proyecto no está \textbf{integrado completamente} con la aplicación móvil de mi compañero Emilio. Ambas aplicaciones se han construido bajo una misma lista de requisitos y funcionalidades, pero es necesario integrar ambas plataformas bajo una misma API para que lo que se haga en una se refleje en la otra, y viceversa.
    \item Partiendo del punto anterior, en la aplicación web \textbf{faltan diversas funcionalidades} que sí están presentes en la móvil, como es el caso del chat individual entre usuarios.
    \item Aunque la aplicación web presenta un diseño adaptable que permite su visualización en todos los formatos de pantalla, sería interesante estudiar el llevarlo a \textbf{otras plataformas móviles} como iOS o Windows 10 Mobile, de forma nativa.
    \item La \textbf{gestión del contenido multimedia} es mejorable en la aplicación web, pudiendo implementar una funcionalidad similar a la existente en la aplicación móvil.
    \item Se pueden realizar \textbf{mejoras en las funcionalidades existentes} actualmente. Los proyectos pueden tener más complejidad o información que mostrar al usuario, así como mejoras en la funcionalidad de vacantes y avances.
    \item La aplicación web contiene \textbf{soporte para múltiples idiomas}. Aunque de momento solo cuenta con el inglés y el español, siempre se pueden añadir más para que más personas puedan hacer uso de SmartU.
\end{itemize}
