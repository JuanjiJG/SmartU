% Definición de la clase de documentación
\documentclass[a4paper, 11pt]{book} % A4 paper and 11pt font size

% Entrada y salida de texto
\usepackage[T1]{fontenc}
\usepackage[utf8]{inputenc}
\usepackage[sfdefault]{roboto}  %% Option 'sfdefault' only if the base font of the document is to be sans serif

% Idioma
\usepackage[spanish, es-tabla]{babel} % Selecciona el español para palabras introducidas automáticamente, p.ej. "septiembre" en la fecha y especifica que se use la palabra Tabla en vez de Cuadro

% Información reutilizable
\newcommand{\asunto}{Trabajo de Fin de Grado}
\newcommand{\titulo}{Análisis de plataforma de recursos de apoyo a la docencia Prado2}
\newcommand{\tituloEng}{Analysis of teaching support resources platform Prado2}
\newcommand{\grado}{Grado en Ingeniería Informática}
\newcommand{\autor}{Juan José Jiménez García}
\newcommand{\email}{juanjojg@correo.ugr.es}
\newcommand{\tutor}{Miguel Gea Megías}
\newcommand{\escuela}{Escuela Técnica Superior de Ingenierías Informática y de Telecomunicación}
\newcommand{\universidad}{Universidad de Granada}
\newcommand{\ciudad}{Granada}
\newcommand{\vers}{Versión 0.1}
\providecommand{\keywords}{software libre}
\providecommand{\keywordsen}{free software}

% Otros paquetes
% \usepackage{amsmath,amsfonts,amsthm} % Math packages
% %\usepackage{graphics,graphicx, floatrow} %para incluir imágenes y notas en las imágenes
% \usepackage{graphics,graphicx, float} %para incluir imágenes y colocarlas
% \graphicspath{{img/}}
%
% \usepackage{xcolor}
% \usepackage{url}
% \usepackage[hidelinks]{hyperref}
%
% % \usepackage[bookmarks=true,
% %             bookmarksnumbered=false, % true means bookmarks in
% %                                      % left window are numbered
% %             bookmarksopen=false,     % true means only level 1
% %                                      % are displayed.
% %             colorlinks=true,
% %             urlcolor=webblue,
% %             linkcolor=webblue]{hyperref}
%
% \definecolor{webgreen}{rgb}{0, 0.5, 0} % less intense green
% \definecolor{webblue}{rgb}{0, 0, 0.5}  % less intense blue
% \definecolor{webred}{rgb}{0.5, 0, 0}   % less intense red
%
% \usepackage{eurosym}
% \usepackage{colortbl}
% \definecolor{Gray}{gray}{0.9}
% %\usepackage{minted}
% \usepackage{fancyvrb}
%
% % Para hacer tablas comlejas
% %\usepackage{multirow}
% %\usepackage{threeparttable}
%
% %\usepackage{sectsty} % Allows customizing section commands
% %\allsectionsfont{\centering \normalfont\scshape} % Make all sections centered, the default font and small caps
%
% \usepackage{lastpage} % Para poder hacer uso del número de la última página
% \usepackage{fancyhdr} % Custom headers and footers
% \pagestyle{fancyplain} % Makes all pages in the document conform to the custom headers and footers
% \fancyhead{} % No page header - if you want one, create it in the same way as the footers below
% \fancyfoot[L]{} % Empty left footer
% \fancyfoot[C]{Página \thepage \hspace{1pt} de \pageref{LastPage}} % Page numbering for center footer
% \fancyfoot[R]{} % Empty right footer
% \renewcommand{\headrulewidth}{0pt} % Remove header underlines
% \renewcommand{\footrulewidth}{0pt} % Remove footer underlines
% \setlength{\headheight}{13.6pt} % Customize the height of the header
%
% \numberwithin{equation}{section} % Number equations within sections (i.e. 1.1, 1.2, 2.1, 2.2 instead of 1, 2, 3, 4)
% \numberwithin{figure}{section} % Number figures within sections (i.e. 1.1, 1.2, 2.1, 2.2 instead of 1, 2, 3, 4)
% \numberwithin{table}{section} % Number tables within sections (i.e. 1.1, 1.2, 2.1, 2.2 instead of 1, 2, 3, 4)
%
% \setlength\parindent{0pt} % Removes all indentation from paragraphs - comment this line for an assignment with lots of text
%
% \newcommand{\horrule}[1]{\rule{\linewidth}{#1}} % Create horizontal rule command with 1 argument of height
