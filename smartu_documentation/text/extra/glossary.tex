\chapter{Glosario de términos}

\begin{description}
    \item[Equipo multidisciplinar] Conjunto de personas, con diferentes formaciones académicas y experiencias profesionales, que operan en conjunto, durante un tiempo determinado, abocados a resolver un problema complejo, es decir, que tienen un objetivo común. Cada individuo es consciente de su papel y del papel de los demás, y trabajan en conjunto bajo la dirección de un coordinador.
    \item[StartUp] Empresa emergente que busca arrancar, emprender o montar un negocio, generalmente apoyada en la tecnología para su desarrollo. Son ideas que innovan el mercado y buscan facilitar los procesos complicados, enfocadas a diferentes temas y usos. Generalmente son empresas asociadas a la innovación, al desarrollo de tecnologías, al diseño web o al desarrollo web.
    \item[Sinergia] La sinergia es una propiedad inherente de los sistemas que establece que las interacciones entre las partes o componentes de un sistema generan un valor agregado mayor al que se lograría si cada componente funcionara por separado.\\
    Aplicado al trabajo en equipo, surgen sinergias cuando la colaboración entre miembros produce más resultados que haciéndo cada uno su parte sin contar con el otro.
    \item[Framework de desarrollo] Es una estructura conceptual y tecnológica de asistencia definida, normalmente, con artefactos o módulos concretos de software, que puede servir de base para la organización y desarrollo de software. Típicamente, puede incluir soporte de programas, bibliotecas, y un lenguaje interpretado, entre otras herramientas, para así ayudar a desarrollar y unir los diferentes componentes de un proyecto.
    \item[Diseño adaptable] El diseño adaptable establece que un sistema informático debe ser reactivo a la interacción del usuario con el mismo y al entorno en el que se encuentra. Esto quiere decir que ha de ``adaptarse'' a todo tipo de dispositivos donde se esté ejecutando o visualizando.
    \item[Diseño \textit{Mobile First}] es una filosofía de diseño que establece que todo desarrollo debe centrar su diseño en visualizar cómo se vería en un dispositivo móvil, y a partir de ese punto ampliar el tamaño de la pantalla e ir reorganizando los elementos.\\
    Por lo general, el diseño \textit{mobile first} comienza apilando uno encima de otro los elementos de la interfaz, y a medida que se gana espacio horizontal, colocar en este los elementos que estaban al fondo.
\end{description}
